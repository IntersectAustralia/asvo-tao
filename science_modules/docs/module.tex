\title{Module Design}
\author{
  Luke Hodkinson \\
  Center for Astrophysics and Supercomputing \\
  Swinburne University of Technology \\
  Melbourne, Hawthorn 32000, \underline{Australia}
}
\date{\today}

\documentclass[14pt]{article}
\usepackage{color}
\usepackage[usenames,dvipsnames]{xcolor}
\usepackage{amsmath}
\usepackage{amsfonts}
\usepackage{listings}
%% \usepackage[scaled]{beramono}
%% \renewcommand*\familydefault{\ttdefault}
%% \usepackage[Tl]{fontenc}

\newcommand{\deriv}[2]{\ensuremath{\frac{\mathrm{d}#1}{\mathrm{d}#2}}}
\newcommand{\sderiv}[2]{\ensuremath{\frac{\mathrm{d}^2#1}{\mathrm{d}#2^2}}}

%% \lstset{
%%   language=Python,
%%   showstringspaces=false,
%%   formfeed=\newpage,
%%   tabsize=4,
%%   basicstyle=\small\ttfamily,
%%   commentstyle=\color{BrickRed}\itshape,
%%   keywordstyle=\color{blue},
%%   stringstyle=\color{OliveGreen},
%%   morekeywords={models, lambda, forms, dict, list, str, import, dir, help,
%%    zip, with, open}
%% }
\lstset{
  language=C++,
  showstringspaces=false,
  formfeed=\newpage,
  tabsize=4,
  basicstyle=\small\ttfamily,
  commentstyle=\color{BrickRed}\itshape,
  keywordstyle=\color{blue},
  stringstyle=\color{OliveGreen}
  %% morekeywords={models, lambda, forms, dict, list, str, import, dir, help,
  %%  zip, with, open}
}

\begin{document}
\maketitle

\section{Introduction}

The science modules form a core part of the TAO system, not only from
an operational sense but also as a part of continued development. It
is our responsibility to ensure future developers are able to
modify existing modules and create new ones with a minimum of
difficulty.

This document describes the design of the science modules'
framework to allow for intuitive development. There are two major
sections, the first describes the interface for modifying and
developing new modules, and the second section describes how
new modules may be added into the primary TAO binary.

\section{Science Module API}

The science module API can be broken down into N subsections:
\begin{itemize}
  \item module processing,
  \item module parameters,
  \item data pipeline,
  \item execution ordering, and
  \item XML description.
\end{itemize}
Each of these subsections will be discussed, in order, in what follows.
Prior to that is provided a UML diagram of the overall architecture
of the API as a reference in \ref{overall_uml}.

\subsection{Module Processing}

The question to be answered here is, ``where do I add my code to perform
the processing my custom module foo needs to perform?'' The base level
module provides a virtual method named \lstverb!execute( tao::galaxy& galaxy )!.
This is the primary interface for performing the execution of each module.
\lstverb!execute! will be called to perform the module's processing on
the current chunk of galaxy data at the appropriate times by the TAO
system.

\subsection{Module Parameters}

Talk about the dictionary.

\subsection{Data Pipeline}

\subsubsection{Passing Data}

Passing data from one module to the next is made a little tricky as
a result of needed to pass data from the database alongside data
generated from other modules. Once data has been acquired from the
database we cannot then insert new columns into the returned rows.
To handle this we provide a \lstverb!galaxy! data structure designed
to provide a consistent interface for setting and retrieving named
data to be made available to other modules. The \lstverb!galaxy!
object is the accepted, and only provided, mechanism for passing
data between modules.

\subsubsection{Data Chunking}

To minimise inefficiency data may be set on the \lstverb!galaxy!
object as vector data. It is expected that all data will be set
as vector data, even if it is of length one, to generalise the
interface for science modules. This way, a science module should
be designed to work with arbitrarily sized vectors of input
data.

To facilitate setting and retrieving vectored data, several methods
are provided on the galaxy object.

\section{Execution Ordering}

The science module execution is heavily dependent on ordering. There is no
sense in executing the SED module before the lightcone module has
executed. To facilitate ordering a directed acyclic graph approach is used.
Each module tracks a set of its parents (\ref{parents_uml}) and ensures that
each parent has been executed before it allows itself to execute. See
figure \ref{execute_dag} for a code snippet on how this operates.

In the case of data chunking where the chunk size is less than the total number
of galaxies generated for the local process, the overall execution loop will
need to be executed multiple times, once for each chunk. This raises the
question, how does the execution system allow each module to know it needs
to run again, even if it has already run? To provide this functionality we
use a simple counter to track which chunk iteration we are currently working on.
A module stores the index of the counter for the most recently processed chunk.
It will not execute again until the counter it sees is greater than its
stored value.

Terminating the overall execution is handled by each module keeping track
of an internal \lstverb!complete! boolean flag. Any module can set this
flag itself as a part of its operation, or it can wait for this flag to
be set by the TAO system automatically when all of the module's parents
report that they are complete.

\begin{figure}
  \begin{lstlisting}
// Process all parents.
bool all_complete = !_parents.empty();
for( auto& parent : _parents )
{
  parent->process( iteration );
  if( !parent->complete() )
    all_complete = false;
}

// Call the user-defined execute routine.
if( !all_complete )
  execute();
else
  _complete = true;
  \end{lstlisting}
\end{figure}

\begin{figure}
  \begin{lstlisting}
// Keep looping over modules until all report being complete.
bool complete;
unsigned long long it = 1;
do
{
  // Reset the complete flag.
  complete = true;

  // Loop over the modules.
  for( auto module : tao::factory )
  {
    module->process( it );
    if( !module->complete() )
      complete = false;
    }

    // Advance the counter.
    ++it;
  }
while( !complete );
  \end{lstlisting}
\end{figure}

\end{document}
