\title{SED Equations}
\author{
  Luke Hodkinson \\
  Center for Astrophysics and Supercomputing \\
  Swinburne University of Technology \\
  Melbourne, Hawthorn 32000, \underline{Australia}
}
\date{\today}

\documentclass[12pt]{scrartcl}
\usepackage{color}
\usepackage[usenames,dvipsnames]{xcolor}
\usepackage{amsmath}
\usepackage{amsfonts}
\usepackage{amssymb}
\usepackage{listings}
%% \usepackage[scaled]{beramono}
%% \renewcommand*\familydefault{\ttdefault}
%% \usepackage[Tl]{fontenc}

\newcommand{\deriv}[2]{\ensuremath{\frac{\mathrm{d}#1}{\mathrm{d}#2}}}
\newcommand{\sderiv}[2]{\ensuremath{\frac{\mathrm{d}^2#1}{\mathrm{d}#2^2}}}
\newcommand{\dx}[1]{\ensuremath{\,\mathrm{d}#1}}

%% \lstset{
%%   language=Python,
%%   showstringspaces=false,
%%   formfeed=\newpage,
%%   tabsize=4,
%%   basicstyle=\small\ttfamily,
%%   commentstyle=\color{BrickRed}\itshape,
%%   keywordstyle=\color{blue},
%%   stringstyle=\color{OliveGreen},
%%   morekeywords={models, lambda, forms, dict, list, str, import, dir, help,
%%    zip, with, open}
%% }

\begin{document}
\maketitle

\section{Apparent Magnitude}

\[ I = \int_\nu f_\nu s_\nu \dx{\nu} \]
$I$ is the intensity of the object, $f_\nu$ is the spectral energy density of
the object in $erg\cdot s^{-1}\cdot Hz^{-1}$ and $s_\nu$ is the unitless band-pass filter.

\[ f_a = \frac{I}{4\pi r^2} \]
$f_a$ is the flux density, $r$ is the distance to the object in $cm$.

\[ f_b = \int_\nu s_\nu \dx{\nu} \]
$f_b$ is the reference value used in the AB magnitude system.

\[ m_{ab} = -2.5\log_{10}\left(\frac{f_a}{f_b}\right) - 48.60 \]
$m_{ab}$ is the apparent magnitude.

\section{Expressions for $f_\nu$ and $s_\nu$}

Recall the equation for intensity:
\[ I = \int_\nu f_\nu s_\nu \dx{\nu} \]
Notice that this equation requires a definition for $f_\nu$ and $s_\nu$,
the spectral energy density and the transmission function, repsectively, each
defined on the frequency domain. We have values for $f_\lambda$ and $s_\lambda$,
but in general we are unable to directly equate the $\lambda$ versions with
the $\nu$ ones. However, we know that the energy of these values
must be equal for any arbitrary subdomain,
\[ f_\nu \dx{\nu} = f_\lambda \dx{\lambda} \; , \]
which can be rearranged to produce
\[ f_\nu = f_\lambda \frac{\dx{\lambda}}{\dx{\nu}} \; . \]
Now, we only need an expression for $\frac{\dx{\lambda}}{\dx{\nu}}$
to complete the transformation. We can use
\begin{eqnarray*}
\lambda & = & \frac{c}{\nu} \\
\therefore \;\;\;\; \frac{\dx{\lambda}}{\dx{\nu}} & = & -\frac{c}{\nu^2} \; ,
\end{eqnarray*}
and substituting into the previous equation yields
\[ f_\nu = -f_\lambda \frac{c}{\nu^2} \; . \]
We can also substitute the relationship between $\lambda$ and $\nu$ into
the right hand side if convenient for the software,
\[ f_\nu = -f_\lambda \frac{\lambda^2}{c} \; . \]

In addition to converting the spectra function from the wavelength domain
to the frequency domain, we must perform the same procedure for the
transmission function, which we initially have in the wavelength domain
only. Beginning with
\[ s_\nu \dx{\nu} = s_\lambda \dx{\lambda} \]
and following the same method as above we end with
\[ s_\nu = -s_\lambda \frac{\lambda^2}{c} \; . \]
Substituting both of these equations into the equation for intensity
we get
\[ I = \int_\nu f_\lambda s_\lambda\frac{\lambda^4}{c^2}\dx{\nu} \; . \]

We could end here, but we can save a little computation by now transforming
this final equation into the wavelength domain,
\begin{eqnarray*}
I & = & \int_\nu f_\lambda s_\lambda\frac{\lambda^4}{c^2}\dx{\nu} \\
& = & -\int_\lambda f_\lambda s_\lambda\frac{\lambda^4}{c^2} \frac{c}{\lambda^2} \dx{\lambda} \\
& = & -\int_\lambda f_\lambda s_\lambda\frac{\lambda^2}{c} \dx{\lambda} \; .
\end{eqnarray*}
Note that the negative sign out the front of the RHS is only there
as when we swap to the $\lambda$ domain we have implicitly reversed the
integration direction. If we are careful to integrate from low to high
terminals we can eliminate it,
\[ I = \int_\lambda f_\lambda s_\lambda\frac{\lambda^2}{c} \dx{\lambda} \; . \]

\end{document}
