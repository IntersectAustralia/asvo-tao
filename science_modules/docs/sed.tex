\title{SED Equations}
\author{
  Luke Hodkinson \\
  Center for Astrophysics and Supercomputing \\
  Swinburne University of Technology \\
  Melbourne, Hawthorn 32000, \underline{Australia}
}
\date{\today}

\documentclass[12pt]{scrartcl}
\usepackage{color}
\usepackage[usenames,dvipsnames]{xcolor}
\usepackage{amsmath}
\usepackage{amsfonts}
\usepackage{amssymb}
\usepackage{listings}
%% \usepackage[scaled]{beramono}
%% \renewcommand*\familydefault{\ttdefault}
%% \usepackage[Tl]{fontenc}

\newcommand{\deriv}[2]{\ensuremath{\frac{\mathrm{d}#1}{\mathrm{d}#2}}}
\newcommand{\sderiv}[2]{\ensuremath{\frac{\mathrm{d}^2#1}{\mathrm{d}#2^2}}}
\newcommand{\dx}[1]{\ensuremath{\,\mathrm{d}#1}}

%% \lstset{
%%   language=Python,
%%   showstringspaces=false,
%%   formfeed=\newpage,
%%   tabsize=4,
%%   basicstyle=\small\ttfamily,
%%   commentstyle=\color{BrickRed}\itshape,
%%   keywordstyle=\color{blue},
%%   stringstyle=\color{OliveGreen},
%%   morekeywords={models, lambda, forms, dict, list, str, import, dir, help,
%%    zip, with, open}
%% }

\begin{document}
\maketitle

\section{Apparent Magnitude}

\[ I = \int_\nu f_\nu s_\nu \dx{\nu} \]
$I$ is the intensity of the object, $f_\nu$ is the spectral energy density of
the object in $erg\cdot s^{-1}\cdot Hz^{-1}$ and $s_\nu$ is the unitless band-pass filter.

\[ f_a = \frac{I}{4\pi r^2} \]
$f_a$ is the flux density, $r$ is the distance to the object in $cm$.

\[ f_b = \int_\nu s_\nu \dx{\nu} \]
$f_b$ is the reference value used in the AB magnitude system.

\[ m_{ab} = -2.5\log_{10}\left(\frac{f_a}{f_b}\right) - 48.60 \]
$m_{ab}$ is the apparent magnitude.

\section{Finding An Expression for $f_\nu$}

Recall the equation for intensity:
\[ I = \int_\nu f_\nu s_\nu \dx{\nu} \]
Notice that this equation requires a definition for $f_\nu$ and $s_\nu$,
the spectral energy density and the transmission function, repsectively, each
defined on the frequency domain. We have values for $f_\lambda$ and $s_\lambda$,
but in general we are unable to directly equate the $\lambda$ versions with
the $\nu$ ones. However, we know that the energy of these values
must be equal for any arbitrary subdomain,
\begin{eqnarray*}
\nu \dx{\nu} & = & \lambda \dx{\lambda} \\
\int_\nu f_\nu \dx{\nu} & = & \int_\lambda f_\lambda \dx{\lambda} \; .
\end{eqnarray*}
and we can use this equation to deduce an appropriate function for
$f_\nu$ and $s_\nu$.

We begin by using the equations
\begin{eqnarray*}
\lambda & = & \frac{c}{\nu} \\
\mathrm{d}{\lambda} & = & -\frac{c}{\nu^2}\dx{\nu}
\end{eqnarray*}
to perform a change of variable of the $\lambda$ form of the energy,
\[ \int_\nu f_\nu \dx{\nu} = -\int_\nu f_\lambda \frac{c}{\nu^2} \dx{\nu} \; . \]
Firstly, we can remove the negative on the RHS; this is a result
of swapping the direction of integration. We will define the integration
directions to be the same, giving
\[ \int_\nu f_\nu \dx{\nu} = \int_\nu f_\lambda \frac{c}{\nu^2} \dx{\nu} \; . \]
Because these integrals can be performed over any arbitrary subdomain of the
frequency space the integrands must be equal, thusly
\begin{eqnarray*}
\int_\nu f_\nu \dx{\nu} & = & \int_\nu f_\lambda \frac{c}{\nu^2} \dx{\nu} \\
\int_\nu f_\nu \dx{\nu} - \int_\nu f_\lambda \frac{c}{\nu^2} \dx{\nu} & = & 0 \\
\int_\nu f_\nu - f_\lambda \frac{c}{\nu^2} \dx{\nu} & = & 0 \\
\therefore \;\;\; f_\nu & = & f_\lambda \frac{c}{\nu^2} \; .
\end{eqnarray*}
We can also substitute the relationship between $\lambda$ and $\nu$ into
the right hand side if convenient for the software,
\[ f_\nu = f_\lambda \frac{\lambda^2}{c} \; . \]

In addition to converting the spectra function from the wavelength domain
to the frequency domain, we must perform the same procedure for the
transmission function, which we initially have in the wavelength domain
only. Beginning with
\[ \int_\nu s_\nu \dx{\nu} = \int_\lambda s_\lambda \dx{\lambda} \]
and following the same method as above we end with
\[ s_\nu = s_\lambda \frac{\lambda^2}{c} \; . \]

Substituting both of these equations into the equation for intensity
we get
\[ I = \int_\nu f_\lambda s_\lambda\frac{\lambda^4}{c^2}\dx{\nu} \; . \]

\end{document}
