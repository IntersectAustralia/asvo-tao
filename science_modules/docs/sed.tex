\title{SED Equations}
\author{
  Luke Hodkinson \\
  Center for Astrophysics and Supercomputing \\
  Swinburne University of Technology \\
  Melbourne, Hawthorn 32000, \underline{Australia}
}
\date{\today}

\documentclass[12pt]{scrartcl}
\usepackage{color}
\usepackage[usenames,dvipsnames]{xcolor}
\usepackage{amsmath}
\usepackage{amsfonts}
\usepackage{listings}
%% \usepackage[scaled]{beramono}
%% \renewcommand*\familydefault{\ttdefault}
%% \usepackage[Tl]{fontenc}

\newcommand{\deriv}[2]{\ensuremath{\frac{\mathrm{d}#1}{\mathrm{d}#2}}}
\newcommand{\sderiv}[2]{\ensuremath{\frac{\mathrm{d}^2#1}{\mathrm{d}#2^2}}}
\newcommand{\dx}[1]{\ensuremath{\,\mathrm{d}#1}}

%% \lstset{
%%   language=Python,
%%   showstringspaces=false,
%%   formfeed=\newpage,
%%   tabsize=4,
%%   basicstyle=\small\ttfamily,
%%   commentstyle=\color{BrickRed}\itshape,
%%   keywordstyle=\color{blue},
%%   stringstyle=\color{OliveGreen},
%%   morekeywords={models, lambda, forms, dict, list, str, import, dir, help,
%%    zip, with, open}
%% }

\begin{document}
\maketitle

\section{Apparent Magnitude}

\[ I = \int_\nu f_\nu s_\nu \dx{\nu} \]
$I$ is the intensity of the object, $f_\nu$ is the spectral energy density of
the object in $erg\cdot s^{-1}\cdot Hz^{-1}$ and $s_\nu$ is the unitless band-pass filter.

\[ f_a = \frac{I}{4\pi r^2} \]
$f_a$ is the flux density, $r$ is the distance to the object in $cm$.

\[ f_b = \int_\nu s_\nu \dx{\nu} \]
$f_b$ is the reference value used in the AB magnitude system.

\[ m_{ab} = -2.5\log_{10}\left(\frac{f_a}{f_b}\right) - 48.60 \]
$m_{ab}$ is the apparent magnitude.

\section{Finding An Expression for $f_\nu$}

Recall the equation for intensity:
\[ I = \int_\nu f_\nu s_\nu \dx{\nu} \]
Notice that this equation requires a definition for $f_\nu$, the spectral
energy density defined on the frequency domain. To be more specific, we
need an expression for
\[ f_\nu = f_\nu(\nu) \;. \]
We have values for $f_\lambda$, but in general these two functions are
not equal. However, the energy of each function must be equal,
\[ \int_\nu f_\nu(\nu) \dx{\nu} = \int_\lambda f_\lambda(\lambda) \dx{\lambda} \;, \]
and we can use this equation to deduce an appropriate function for
$f_\nu$. We begin by using the equations
\begin{eqnarray*}
\lambda & = & \frac{c}{\nu} \\
\mathrm{d}{\lambda} & = & -\frac{c}{\nu^2}\dx{\nu}
\end{eqnarray*}
to perform a change of variable of the $\lambda$ form of the energy,
\[ -\int_\nu f_\lambda\left(\frac{c}{\nu}\right) \frac{c}{\nu^2} \dx{\nu} \; . \]
We know this form must be equatable to the $\nu$ energy equation
above, meaning we can elicit that
\[ f_\nu(\nu) = -f_\lambda\left(\frac{c}{\nu}\right) \frac{c}{\nu^2} \]
In the code we would prefer to integrate over $\lambda$ instead of $\nu$.
We can substitute the definition of $\lambda$ into the above equation,
\begin{eqnarray*}
f_\nu(\nu) & = & -f_\lambda(\lambda) \frac{c}{\left(\frac{c}{\lambda}\right)^2} \\
          & = & -f_\lambda(\lambda) \frac{\lambda^2}{c} \\
          & = & f_\nu(\lambda) \; .
\end{eqnarray*}
Using this we can substitute for an expression of
the intensity,
\[ I = -\int_\lambda f_\lambda(\lambda)s(\lambda)\frac{\lambda^2}{c}\dx{\lambda} \; . \]

It's woth noting that the negative appearing at the front of the above equation
is there as a result of the change of variable from $\nu$ to $\lambda$, which
changes the direction of integration. If we are sure to integrate from lower to
higher values of $\lambda$ we can remove it,
\[ I = \int_\lambda f_\lambda(\lambda)s(\lambda)\frac{\lambda^2}{c}\dx{\lambda} \; . \]

\end{document}
