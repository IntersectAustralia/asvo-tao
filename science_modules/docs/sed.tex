\title{SED Equations}
\author{
  Luke Hodkinson \\
  Center for Astrophysics and Supercomputing \\
  Swinburne University of Technology \\
  Melbourne, Hawthorn 32000, \underline{Australia}
}
\date{\today}

\documentclass[12pt]{scrartcl}
\usepackage{color}
\usepackage[usenames,dvipsnames]{xcolor}
\usepackage{amsmath}
\usepackage{amsfonts}
\usepackage{amssymb}
\usepackage{siunitx}
\usepackage{listings}
%% \usepackage[scaled]{beramono}
%% \renewcommand*\familydefault{\ttdefault}
%% \usepackage[Tl]{fontenc}

\newcommand{\deriv}[2]{\ensuremath{\frac{\mathrm{d}#1}{\mathrm{d}#2}}}
\newcommand{\sderiv}[2]{\ensuremath{\frac{\mathrm{d}^2#1}{\mathrm{d}#2^2}}}
\newcommand{\dx}[1]{\ensuremath{\,\mathrm{d}#1}}

%% \lstset{
%%   language=Python,
%%   showstringspaces=false,
%%   formfeed=\newpage,
%%   tabsize=4,
%%   basicstyle=\small\ttfamily,
%%   commentstyle=\color{BrickRed}\itshape,
%%   keywordstyle=\color{blue},
%%   stringstyle=\color{OliveGreen},
%%   morekeywords={models, lambda, forms, dict, list, str, import, dir, help,
%%    zip, with, open}
%% }

\begin{document}
\maketitle

\section{Apparent Magnitude}

We are primarily concerned with calculating magnitudes from the
SED of objects in a lightcone. In order to get started I will
focus on AB magnitudes. The AB magnitude of an object is defined
as
\[ m_{ab} = -2.4\log_{10}F_{\nu} - 48.6 \; , \]
where $F_\nu$ is the spectral flux density in $erg\cdot s^{-1}\cdot cm^{-1} \cdot Hz^{-1}$.
We will be defining $F_\nu$ as
\[ F_\nu = \frac{I}{R} \; , \]
where $I$ is the intensity of the object and $R$ is the normalisation
with respect to the bandpass filter used. $I$ and $R$ are defined
as
\begin{eqnarray*}
I & = & \int_\nu f_\nu r \dx{\nu} \\
R & = & \int_\nu r \dx{\nu}
\end{eqnarray*}
where $f_\nu$ is the sepctral energy density of the object in
$erg\cdot s^{-1}\cdot cm^{-2}\cdot Hz^{-1}$ and $r$ is a unitless transmission
response filter.

The spectral energy densitiy will be given
to us not as a function of frequency, as we need, but instead as
a function of wavelength and in units of
$erg\cdot s^{-1}\cdot cm^{-2}\cdot \angstrom^{-1}$. We refer to this quantity as
$f_\lambda$. In general $f_\lambda \ne f_\nu$, so we
will need to find $f_\nu$ in terms of $f_\lambda$.
We know that the energy of these values
must be equal for any arbitrary subdomain,
\[ f_\nu \dx{\nu} = f_\lambda \dx{\lambda} \; , \]
which can be rearranged to produce
\[ f_\nu = f_\lambda \frac{\dx{\lambda}}{\dx{\nu}} \; . \]
Now, we only need an expression for $\frac{\dx{\lambda}}{\dx{\nu}}$
to complete the transformation. We can use
\begin{eqnarray*}
\lambda & = & \frac{c}{\nu} \\
\therefore \;\;\;\; \frac{\dx{\lambda}}{\dx{\nu}} & = & -\frac{c}{\nu^2} \; ,
\end{eqnarray*}
and substituting into the previous equation yields
\[ f_\nu = -f_\lambda \frac{c}{\nu^2} \; . \]
Substitution into the equation for intensity gives
\[ I = -\int_\nu f_\lambda r \frac{c}{\nu^2} \dx{\nu} \; . \]
We would prefer to integrate over $\lambda$, for conenience, so
we will transform the integral from $\nu$ using the relationship
between $\nu$ and $\lambda$,
\begin{eqnarray*}
I & = & -\int_\lambda f_\lambda r \frac{c}{\nu^2} \left(-\frac{\nu^2}{c}\right) \dx{\lambda} \\
& = & \int_\lambda f_\lambda r \dx{\lambda} \; .
\end{eqnarray*}
This result is as expected; the energy of the filtered SED is the same
irrespective of whether considered on the wavelength or the
frequency domain. It is also very convenient, as we have no need
to transform any quantites to frequency.

While we could leave the expression for $R$ in terms of
frequency, it's more convenient for the code to integrate it in
terms of wavelength. To do this we can simply perform a variable
substitution in the integral,
\begin{eqnarray*}
R & = & \int_\nu r \dx{\nu} \\
& = & \int_\lambda r \left(-\frac{c}{\lambda^2}\right) \dx{\lambda} \\
& = & -\int_\lambda r \frac{c}{\lambda^2} \dx{\lambda} \\
& = & \int_\lambda r \frac{c}{\lambda^2} \dx{\lambda} \; .
\end{eqnarray*}
Note that we are able to eliminate the negation of the RHS because
it appears only as a result of reversing the integration direction.

Now we have all the necessary expressions to define $F_\nu$ in terms
of wavelength,
\begin{eqnarray*}
F_\nu & = & \frac{I}{R} \\
& = & \frac{{\displaystyle \int_\lambda f_\lambda r \dx{\lambda}}}{{\displaystyle \int_\lambda r \frac{c}{\lambda^2} \dx{\lambda}}} \; ,
\end{eqnarray*}
allowing us to provide the full equation for the AB magnitude of an
object:
\[ m_{ab} = -2.5\log_{10} \frac{{\displaystyle \int_\lambda f_\lambda r \dx{\lambda}}}{{\displaystyle \int_\lambda r \frac{c}{\lambda^2} \dx{\lambda}}} - 48.6 \]

\end{document}
